\section{Introduction}

This document is intended as a user's manual for the TVLA system.
The reader should be familiar with the Three-Valued Logic based
Analysis framework described in~\cite{TOPLAS:SRW02} before
consulting this manual.  The manual is accompanied by an example
of the analysis of the reverse function in \figref{Reverse}.

The original algorithms in the system were designed and implemented
by Tal Lev-Ami~\cite{kn:TalSAS00,Master:LevAmi00}.

\subsection{Downloading and installing}

The system and latest information is available at:\\
\url{http://www.cs.tau.ac.il/~tvla/}.\\
Please see the file LICESNE for licensing information.

Installation procedure:
\begin{enumerate}
\item TVLA is written in Java and requires J2SE version 1.6 (or
above), available from \url{http://java.sun.com/j2se/}. Before
attempting to run TVLA, make sure it is possible to launch Java
executables by entering ``java'' at the command-line prompt.

\item TVLA uses DOT to generate Postscript files and requires
Graphviz version 2.24 (or above), available from\\
\url{http://www.research.att.com/sw/tools/graphviz/}.  After
installing Graphviz, make sure the bin sub-directory is added to
your path.

\item Extract the archive's content and set the environment
variable TVLA\_HOME to that location.

\item Add the bin sub-directory to your path (the bin directory
contains running scripts for Windows and Linux).

\textbf{IMPORTANT:} Make sure the path does not contain trailing
'$\backslash$' characters or trailing `/` characters.

\item You are now ready to run the system.  Enter tvla to see
usage information and command-line options.
\end{enumerate}

%\subsection{Summary of this document}
