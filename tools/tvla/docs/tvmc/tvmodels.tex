\section{Three-Valued Models}

A three-valued model defines the behavior of the program to be
verified. It defines the effect of \emph{actions} and how actions
are applied to configurations. A three-valued model consists of
the following:
\begin{itemize}
\item Declarations
\item Action Definitions
\item Thread Types Definitions
\item Properties
\item Output Modifiers
\end{itemize}

\begin{figure}
\begin{center}
\framebox{ \vbox{
\begin{alltt}
\begin{tabbing}

/*** \\
** mutex.tvm \\
** mutual exclusion examples for unbound number of threads. \\
** based on the example from "symmetry and model checking" \\
***/ \\


/*********************** Sets ******************************/ \\
\%s FieldsAndParameters \{ \} \\
\%s Globals \{ aLock \} \\

/**************** Predicates **************/ \\

\#include "con\_pred.tvm" \\
\#include "shape\_pred.tvm" \\

\%p \propertyoccurred() \\

\seperator \\
/********************* Actions *******************/ \\

\#include "con\_stat.tvm" \\

\%action verifyProperty() \{ \\
    \tvprecond  !\propertyoccurred() \\
    \tvmessage\= (E(t\_1,t\_2) (t\_1 != t\_2) \& at[gl\_2](t\_1) \& at[gl\_2](t\_2)) \+ \\
    ->  "mutual exclusion may be violated" \- \\
\} \\

/**************** Program *********************/ \\

\seperator \\

/***********************************************/ \\
/**************  Threads ****************/ \\

\%thread \= main \{ \+ \\                     \=
        $gl_1$ blockLockGlobal(aLock) $gl_1$        \` // lock(aLock) \\
        $gl_1$ succLockGlobal(aLock) $gl_2$         \`              \\
        $gl_2$ skip() $gl_3$                        \`              \\
        $gl_3$ unlockGlobal(aLock) $gl_4$           \` // unlock(aLock) \- \\
\} \\

\seperator \\
/**********************************************/ \\
/************** Claims        ****************/ \\
verifyProperty() \\

\end{tabbing}
\end{alltt}
} }
\end{center}
\caption{\label{Fi:TmutexTVM}TVM file for mutual exclusion
example.}
\end{figure}


\figref{TmutexTVM} shows the TVM file for the mutual exclusion
example program.

\subsection{Declarations}

The declarations section of a three-valued model may consist of
declarations of \emph{core predicates}, \emph{intrumentation
predicates}, sets and consistency rules. The reader is referred to
\cite{kn:TalSAS00} for more details.

$\tvmc$ automatically defines the predicates above the two-line
separator of \tableref{Predicates}. In addition, $\tvmc$ defines
the set $Labels$ which contains all labels in the program.


\subsection{Action Definitions}

Informally, an action is characterized by the following kinds of
information:
\begin{itemize}
\item A {\em precondition\/} under which the action is \emph{enabled} expressed
as logical formula. This formula may also include a designated
free variable $t_r$ to denote the ``scheduled'' thread on which
the action is performed. Our operational semantics is
non-deterministic in the sense that many actions can be enabled
simultaneously and one of them is chosen for execution. In
particular, it selects the scheduled thread by an assignment to
$t_r$.

\item An \emph{enabled} action creates a new configuration where the
interpretations of every predicate $p$ of arity $k$ is determined
by evaluating a formula $\varphi_p(v_1, \ldots, v_k)$ which may
use $v_1, \ldots, v_k$ and $t_r$ as well as all other predicates
in $P$.
\end{itemize}

A program statement may be modeled by several alternative actions
corresponding to the different behaviors of the statement. A
single action to be taken is determined by evaluation of the
preconditions.

Technically, an action consists of the following:
\begin{itemize}
\item Title - ($\tvtitle$) a textual title for the action.
\item Focus formulae ($\tvfocus$) - focus formulae for the action, applied
before the precondition is evaluated.
\item Precondition formula ($\tvprecond$) - evaluated to determine if the action
is \emph{enabled} (could be applied) for a given configuration.
The precondition formula may have free variables, in which case,
the action is applied for every assignment that may satisfy the
formula. A special free variable $t_r$ denotes the ``scheduled''
thread on which the action is performed.
\item Universe modifiers ($\new$, $\newthread$, $\retain$) - universe modifiers allow to add/remove
individuals and thread-individuals to the universe of the
configuration.
\item Thread state modifiers ($\tstart$, $\tstop$)- allow to start and stop a thread.
An unary formula must be supplied to identify threads to be
affected by the modifier.
\item Analysis Control ($\tvhalt$) - halts the analysis.
\item Update formulae - update the configuration to model the
effect of the action.
\end{itemize}



\subsection{Thread Types}

$\tvmc$ allows the user to define thread-types. A thread-type is
defined using the keyword \texttt{\%thread} followed by the
thread-type name and the control-flow for the thread-type. A
thread-type can be later instantiated as explained in the
following section.

\figref{ThreadDef} shows a definition of a thread-type named
\emph{main} which consists of 4 actions.

\begin{figure}
\begin{center}
\framebox{ \vbox{
\begin{alltt}
\begin{tabbing}

\%thread \= main \{ \+ \\                     \=
        $gl_1$ blockLockGlobal(aLock) $gl_1$        \` // lock(aLock) \\
        $gl_1$ succLockGlobal(aLock) $gl_2$         \`              \\
        $gl_2$ skip() $gl_3$                        \`              \\
        $gl_3$ unlockGlobal(aLock) $gl_4$           \` // unlock(aLock) \- \\
\} \\

\end{tabbing}
\end{alltt}
} }
\end{center}
\caption{\label{Fi:ThreadDef}Definition of a thread type.}
\end{figure}


\subsubsection{Thread Control-flow}

Control flow in a thread-type is expressed in terms of
transitions. A transition consists of a source-label, an action,
and a target-label. CFG nodes are implicitly defined for each
label when a transition using these labels is defined. $\tvmc$
supports non-determinism when more than one enabled transition
exists from a single CFG node.

$\tvmc$ assumes that threads are scheduled arbitrarily, and thus
thread actions may be arbitrarily interleaved with actions of
other threads. A special keyword \textbf{atomic} may be used to
define atomic blocks as explained in Section~\ref{Se:Atomic}.

The values of the predicates $at[lab](tr)$ are automatically
updated when an action is applied to express the change of
$at[lab](tr)$ according to the transition. It is possible to
disable automatic update of $at[lab](tr)$ predicates by using the
keyword \tvexplicitat in the action definition header. When
explicit update is enable (automatic update disabled), the user
should provide predicate-update formulae for $at[lab](tr)$ as part
of the action's update formulae.

\subsubsection{Instantiation of Threads}

A new thread can be instantiated using the universe-modifier
$\newthread$. The universe modifier $\newthread$ takes a name of a
thread-type as parameter and create a new thread individual of the
given type. The predicate $isNew(v)$ evaluates to true for the
newly created thread node. The predicate $isNew(v)$ can be used by
the action's update formulae. A created thread is initially not
ready for scheduling.

\subsubsection{Thread Scheduling}

The thread-state modifiers $\tstart$ and $\tstop$ allow to add
thread control in an action. The modifiers require an unary
formula to identify threads to be affected by the modifier. The
unary predicate $ready(v)$ evaluates to $true$ for thread
individuals that are ready for scheduling.

\subsubsection{Analysis Control}

The $\tvhalt$ modifier allows an action to be used to halt the
analysis. When an action with the $\tvhalt$ modifier is applied,
the analysis terminates \emph{immediately}.


\subsection{Atomic Blocks}\label{Se:Atomic}

The keyword \textbf{atomic} is used to define a sequence of
actions that should be performed atomically. An atomic block is
defined using the atomic keyword followed by the block in curly
braces. An atomic block should have a single exit.

\subsection{Assertions}

The keywords \textbf{\%assert} and \textbf{hardassert} can be used
to define assertion statements. If the formula $f$ supplied to
\textbf{\%assert(f)} evaluates to \emph{true} or \emph{unknown},
the statement has no effect. If the formula evaluates to
\emph{false}, the statement will yield an error. The
\textbf{\%hardassert(f)} statement is similar to
\textbf{\%assert(f)}, but yields an error when $f$ evaluates to
\emph{unknown} in addition to \emph{false}.

Note that assertion statements simply save the need for
declaration of \emph{halting actions}.


\subsection{Properties}

Properties are implemented more generally as \emph{global
actions}. Global actions are evaluated on every step of the
analysis regardless of current active threads, current location
and current thread scheduling. A global action may perform any
required action, but it is common to use it only for evaluating a
required property.

\subsection{Output Modifiers}

The output modifiers section allows the user to define which
configurations are written to the output file. The keywords
$\tvinclude$ and $\tvexclude$ are used to define configurations
that should be included in the output or excluded from it.
$\tvinclude$ and $\tvexclude$ take a closed formula as a
parameter. Only configurations for which the formula may evaluate
to \emph{true} are included/excluded. (NYI)
